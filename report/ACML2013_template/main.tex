%\documentclass[wcp,gray]{jmlr} % test grayscale version
\documentclass[wcp]{jmlr}

% The following packages will be automatically loaded:
% amsmath, amssymb, natbib, graphicx, url, algorithm2e

%\usepackage{rotating}% for sideways figures and tables
\usepackage{longtable}% for long tables

% The booktabs package is used by this sample document
% (it provides \toprule, \midrule and \bottomrule).
% Remove the next line if you don't require it.
\usepackage{booktabs}
% The siunitx package is used by this sample document
% to align numbers in a column by their decimal point.
% Remove the next line if you don't require it.
%\usepackage[load-configurations=version-1]{siunitx} % newer version
%\usepackage{siunitx}

% The following command is just for this sample document:
\newcommand{\cs}[1]{\texttt{\char`\\#1}}

\jmlrvolume{29}
\jmlryear{2013}
\jmlrworkshop{ACML 2013}

\title[Short Title]{Full Title of Article}

 % Use \Name{Author Name} to specify the name.
 % If the surname contains spaces, enclose the surname
 % in braces, e.g. \Name{John {Smith Jones}} similarly
 % if the name has a "von" part, e.g \Name{Jane {de Winter}}.
 % If the first letter in the forenames is a diacritic
 % enclose the diacritic in braces, e.g. \Name{{\'E}louise Smith}

 % Two authors with the same address
 % \author{\Name{Author Name1} \Email{abc@sample.com}\and
 %  \Name{Author Name2} \Email{xyz@sample.com}\\
 %  \addr Address}

 % Three or more authors with the same address:
 % \author{\Name{Author Name1} \Email{an1@sample.com}\\
 %  \Name{Author Name2} \Email{an2@sample.com}\\
 %  \Name{Author Name3} \Email{an3@sample.com}\\
 %  \Name{Author Name4} \Email{an4@sample.com}\\
 %  \Name{Author Name5} \Email{an5@sample.com}\\
 %  \Name{Author Name6} \Email{an6@sample.com}\\
 %  \Name{Author Name7} \Email{an7@sample.com}\\
 %  \Name{Author Name8} \Email{an8@sample.com}\\
 %  \Name{Author Name9} \Email{an9@sample.com}\\
 %  \Name{Author Name10} \Email{an10@sample.com}\\
 %  \Name{Author Name11} \Email{an11@sample.com}\\
 %  \Name{Author Name12} \Email{an12@sample.com}\\
 %  \Name{Author Name13} \Email{an13@sample.com}\\
 %  \Name{Author Name14} \Email{an14@sample.com}\\
 %  \addr Address}


 % Authors with different addresses:
  \author{\Name{Author Name1} \Email{abc@sample.com}\\
  \addr Address 1
  \AND
  \Name{Author Name2} \Email{xyz@sample.com}\\
  \addr Address 2
 }

\editor{Cheng Soon Ong and Tu Bao Ho}
% \editors{List of editors' names}

\begin{document}

\maketitle

\begin{abstract}
This paper aims at bringing out the merits of using the Laplace distribution over
the Normal distribution. The choice of the best distribution is objectively
made using the minimum message length (MML) principle. The message length for
transmitting data using a Laplace distribution is derived and its parameters are
estimated. This method of transmission is compared with that of transmistting the data
using the Normal distribution. This is explored in the context of superposition
of protein structures. The optimal superposition of protein structures (described
using a Normal distribution) minimizing the L2 norm is computed using the 
Kearsley's method and the superposition minimizing the
L1 norm (described using a Laplace model) is approximated using Monte Carlo 
simulation. These two are compared with respect
to their model complexity and the overall fit to the data using MML. 
\end{abstract}

\begin{keywords}
Laplace, Normal, MML, Kearsley, Monte Carlo simulation
\end{keywords}

\section{Introduction}
Normal distribution is widely used in modelling a set of data whose true distribution
is unknown. In many problems, the objective function is formulated as a sum of squares,
(the L2 norm) and this function is minimized or maximized depending on the application. 
Normal distribution has a huge impact on the cost function because of the squared nature of 
the individual terms. If there are outliers in the dataset, the final inference might be 
skewed to accommodate the outliers in the model description. The Laplace distribution, 
however, is robust to
outliers as the objective function involves the sum of the absolute values of the difference 
of the individual terms (L1 norm).
The choice of selecting a Laplace over Normal is investigated in this paper. This
selection is made by formulating the objective function using Minimum Message
Length (MML). The distribution which results in the best compression of data is chosen
to be the best model. \\

The general procedure to formulate the message length expression for transmitting
data using some statistical model is outlined in \citet{wallace-87}. The MML method
of estimating parameters for a number of distributions has been well established
\citep{WallaceBook}. This general approach is followed in the derivation of the
message length expression for the Laplace distribution. Although this has not been
done previously, we believe that it has potential benefits. Laplace distribution
is a model of choice in several applications
If the data is generated from a Normal distribution, its approximation using a Normal
distribution would result in a better compression

Examples of protein structures are considered
and they are superposed using Kearsley's transformation. This results in an
orientation of the proteins which minimizes the total least squares of the
corresponding coordinates. A superposition which minimizes the total absolute 
value of the difference of the coordinates is computed using a Monte Carlo 
simulation. The two orientations 

\acks{Acknowledgements should go at the end, before appendices and references.}

\bibliography{references}

\appendix

\section{First Appendix}\label{apd:first}

This is the first appendix.

\section{Second Appendix}\label{apd:second}

This is the second appendix.

\end{document}
